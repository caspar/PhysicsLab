\documentclass{amsart}
\usepackage[margin=3cm]{geometry}                % See geometry.pdf to learn the layout options. There are lots.
\geometry{letterpaper}                   % ... or a4paper or a5paper or ... 
%\geometry{landscape}                % Activate for for rotated page geometry
\usepackage[parfill]{parskip}    % Activate to begin paragraphs with an empty line rather than an indent
\usepackage{float}
\usepackage{graphicx}
\usepackage{amssymb}
\usepackage{epstopdf}
\usepackage{siunitx}
\usepackage{subcaption}

\DeclareGraphicsRule{.tif}{png}{.png}{`convert #1 `dirname #1`/`basename #1 .tif`.png}

\title{      }
\author{Caspar \textsc{Lant}} % Author name

\date{\today} % Date for the report

\begin{document}

\bigskip

\maketitle % Insert the title, author and date
\begin{center}

Intermediate Experimental Physics II\\
\vspace{1.5cm}

\begin{tabular}{l r}

Section: & 002\\
\\
Date Performed: & February 2, 2015 \\ % Date the experiment was performed
Date Due: & Februrary 9, 2015\\
\\
Partner: & Sam P. Meier \\ % Partner names
Professor: & Prof. Andrew Kent\\ 
Instructor: & David Mykytyn % Instructor/supervisor
\end{tabular}
\end{center}
\vspace{50mm}
\pagebreak

\paragraph{\textbf{The Objective} of this week's experiment was to put our vast theoretical knowledge of lenses to application. It is always a remarkable think to see what was once pure abstraction validated though rigorous scientific experimentation. }

\section{Theoretical Background/ Abstract}
\paragraph{Lenses have been of interest to us humans for a long time now. Convex lenses, concave lenses, even planar mirrors have all been the subjects of frenzied study through the ages. This should come as no surprise, given how fantastically useful they are to creatures who experience the world, mainly, through sight. We are mainly interested in what happens to the light from objects in front of the lens}
\begin{equation}
\frac{1}{s_i}+ \frac{1}{s_o} = \frac{1}{f}
\end{equation}
\begin{equation}
\frac{1}{f} = (n-1)\Big(\frac{1}{R_1} - \frac{1}{R_2}\Big)
\end{equation}

\section{Experimental Procedure}
\begin{enumerate}
\item 
\item 
\item 
\item 
\item 
\item 
\item 
\item 
\item 
\item 
\item 
\item 
\item 
\item 
\end{enumerate}

\section{}

\medskip 

\pagebreak

\section{Questions}

\begin{enumerate}
\item {\textit{Question?}
\begin{quote}
Answer
\end{quote}}

\item{\textit{Question?}
\begin{quote}
Answer
\end{quote}}

\end{enumerate}

\section{Error Analysis}




\end{document}