\documentclass{amsart}
\usepackage[margin=3cm]{geometry}                % See geometry.pdf to learn the layout options. There are lots.
\geometry{letterpaper}                   % ... or a4paper or a5paper or ...
%\geometry{landscape}                % Activate for for rotated page geometry
\usepackage[parfill]{parskip}    % Activate to begin paragraphs with an empty line rather than an indent
\usepackage{float}
\usepackage{graphicx}
\usepackage{amssymb}
\usepackage{epstopdf}
\usepackage{siunitx}
\usepackage{subcaption}
\usepackage{units}
\usepackage{setspace}

\DeclareGraphicsRule{.tif}{png}{.png}{`convert #1 `dirname #1`/`basename #1 .tif`.png}

\title{Michelson Interferometer}
\author{Caspar \textsc{Lant}} % Author name

\date{\today} % Date for the report

\begin{document}

\bigskip

\maketitle % Insert the title, author and date
\begin{center}
    Intermediate Experimental Physics II\\
    \vspace{.7cm}
    \begin{tabular}{l r}
        Section: & 002\\
        \\
        Date Performed: & March 29, 2016 \\ % Date the experiment was performed
        Date Due: & April 5, 2016\\
        \\
        Partner: & Neil Saddler\\ % Partner names
        Professor: & Prof. Andrew Kent\\
        Instructor: & David Mykytyn % Instructor/supervisor
    \end{tabular}
    \vfill
    \includegraphics[width=\textwidth]{picture.jpg}
    \vfill
\end{center}

\pagebreak
\setstretch{1.5}
\paragraph{\textbf{The Objective} of this week's experiment was to nullify the claim that light requires a medium of propogation (the so-called ``ether") and to measure the wavelengths of light using our extant knowledge of wave propogation.}

\section{Theoretical Background/ Abstract}
\paragraph{The speed of light in a vaccuum is a universal constant and, according to special relativity, cannot be exceeded. The speed of light is defined to be exactly $299,792,458  \ \nicefrac{m}{s}$. This value is precise because the units of meters and seconds are defined in terms of the speed of light, and not the other way around. If our understanding of the speed of light were to change somehow, or were to become more precise, humanity's standard length and time measurements would shrink or grow accordingly. At first glance, this value is mind-bogglingly large, but it turns out that it is not very difficult to measure, and as a consequence, scientists have had a good idea of the speed of light for a very long time. As I'm sure you know, we physicists often deal with times and distances that far exceed or fall hugely short of this order of magnitude. It is for this reason\-- that we can very precisely measure very small distances and times, that we are so confident in our experimentally-derived value for the speed of light in a vaccuum.\\}
\paragraph{The value of the index of refraction of a given medium is given by the ratio of the speed of light in a vaccuum to the speed of light in that medium. Obviously, this means that the index of refraction of free space is 1, and the refractive indeces of all other media is greater than one. The index of refraction of air is very close to one, and is a value that we will derive further on in this experiment.}
\begin{equation}
    N(t) = N_0 e^{-\lambda t}
\end{equation}
Where $n_m$ is the refractive index of our medium, $\Delta x$, is the distance between the laser and the reflector, and $\l_m$ is the length of our sample medium. In these calculations, we assume that the index of refraction of air is approximately equal to one, the index of refraction in free space.
Rearranging the above definition of the index of refraction, we see that:
\begin{equation}
    T_{\nicefrac{1}{2}} = \dfrac{\ln 2}{\lambda} = \dfrac{0.693}{\lambda}
\end{equation}
Where $c_m$ is the speed of light in our medium, $c_a$ is the speed of light in free space, and $n_m$ is the refractive index of our medium.

\begin{equation}
    \lambda = \dfrac{\ln \left(\nicefrac{N_1}{N_2}\right)}{t_2-t_1}
\end{equation}

\section{Experimental Procedure}
% \setstretch{1.5}
\begin{enumerate}
\item Turn on the Nucleus scalar/timer and connect it to the geiger counter. If the device does not turn on, ensure that it is plugged in
\item Why am I doing this David doesn't read the procedure anyways
\item Most tetrapods have four appendages, as the name implies but there are multiple exceptions to this ``rule", the most salient of whih is the snake!
\item Yes, snakes are tetrapods too! Telling me otherwise would be an insult to all snake-kind.
\item {\tt https://youtube.com/W7qd3cIj78/}
\item That was a totally random sequence of characters. Don't waste your time typing it into your browser to see where I've taken you
\item Maybe it's a funny snake video though...
\item Move the dials on the device to the correct positions. Voltage should not exceed 3V, and the time interval should be 0.1s
\item Using the test samples, pick a distance from the geiger counter that you would like to place your titrated barium solution
\item Attach the length of piping to the syringe and suck up 2ml of the solution in the bottle
\item Remove both caps from the source and squeeze three drops of the solution onto a planchet (new word!)
\item Place the planchet the desired distance from the geiger counter
\item With haste, begin the stopwatch and your first count-interval
\item Repeat this every thirty seconds for a period of 10 minutes.
\item Clean up. Do not eat the solution.
\item Profit!
\end{enumerate}
\setstretch{1.5}
\vfill
\section{Graphs and Tables}

\begin{table}[H]
    \begin{minipage}{0.49\textwidth}
        \centering
        \label{my-label}
        \begin{tabular}{c|c|c}
            Time (s) & Count $\left(\nicefrac{\beta^-}{\rm s}\right)$ & $\delta \mathrm{Count} \left(\nicefrac{\beta^-}{\rm s}\right)$ \\ [1ex]\hline
            0        & 1052                                           & 32.43                                                          \\
            30       & 895                                            & 29.92                                                          \\
            60       & 773                                            & 27.8                                                           \\
            90       & 723                                            & 26.89                                                          \\
            120      & 631                                            & 25.12                                                          \\
            150      & 543                                            & 23.3                                                           \\
            180      & 434                                            & 20.83                                                          \\
            210      & 416                                            & 20.4                                                           \\
            240      & 394                                            & 19.85                                                          \\
            270      & 342                                            & 18.49                                                          \\
            300      & 287                                            & 16.94                                                          \\
            330      & 263                                            & 16.22                                                          \\
            360      & 237                                            & 15.39                                                          \\
            390      & 221                                            & 14.87                                                          \\
            420      & 164                                            & 12.81                                                          \\
            450      & 145                                            & 12.04                                                          \\
            480      & 115                                            & 10.72
        \end{tabular}
    \end{minipage}
    % %
    % \begin{minipage}{0.2\textwidth}
    % \end{minipage}
    % %
    \begin{minipage}{0.49\textwidth}
        \centering
        \label{my-label}
        \begin{tabular}{c|c|c}
            Time (s) & Count $\left(\nicefrac{\beta^-}{\rm s}\right)$ & $\delta \mathrm{Count} \left(\nicefrac{\beta^-}{\rm s}\right)$ \\ [1ex]\hline
            0        & 588                                            & 24.25                                                          \\
            30       & 503                                            & 22.43                                                          \\
            60       & 427                                            & 20.66                                                          \\
            90       & 404                                            & 20.1                                                           \\
            120      & 349                                            & 18.68                                                          \\
            150      & 319                                            & 17.86                                                          \\
            180      & 268                                            & 16.37                                                          \\
            210      & 229                                            & 15.13                                                          \\
            240      & 203                                            & 14.25                                                          \\
             -       &  -                                             &   -                                                            \\
             -       &  -                                             &   -                                                            \\
             -       &  -                                             &   -                                                            \\
             -       &  -                                             &   -                                                            \\
             -       &  -                                             &   -                                                            \\
             -       &  -                                             &   -                                                            \\
             -       &  -                                             &   -                                                            \\
             -       &  -                                             &   -                                                            \\
        \end{tabular}
    \end{minipage}
\end{table}

The uncertainty of our measurement for the index of refraction of air is produced by our error in the distance between the reflector and the laser source. It is equal to 0.016 and is unitless. Our expected value for the index of refraction of air (which of course depends on the density of the air, which depends on the ambient temperature and pressure) is 1.00, which falls within our estimated uncertainty.

\begin{table}[H]
    \begin{minipage}{0.3\textwidth}
        \centering
        \caption{Water}
        \vspace{-0.2cm}
        $L = 51.0\ \unit{cm} \pm 0.1$
        \vspace{0.3cm}
        \label{my-label}
        \begin{tabular}{c|c}
        $d_1 \unit{\ (cm)}$ & $d_2 \unit{\ (cm)} \pm 1.5$ \\ \hline
        49.0            & 81.7                    \\
        48.5            & 84.0                    \\
        49.0            & 72.6                    \\
        48.9            & 81.1                    \\
        48.8            & 80.0                    \\
        48.4            & 87.8
        \end{tabular}
    \end{minipage}
    %
    \begin{minipage}{0.3\textwidth}
        \centering
        \caption{Acryliglas}
        \vspace{-0.2cm}
        $L = 49.0 \ \unit{cm}$
        \vspace{0.3cm}
        \label{my-label}
        \begin{tabular}{c|c}
        $d_1 \unit{\ (cm)}$        & $d_2 \unit{\ (cm)} \pm 1.5$ \\ \hline
        45.0                   & 69.6                    \\
        45.0                   & 68.6                    \\
        45.0                   & 68.5                    \\
        - & - \\
        - & - \\
        - & - \\
        \end{tabular}
    \end{minipage}
    %
    \begin{minipage}{0.3\textwidth}
        \centering
        \caption{Acrylicrap}
        \vspace{-0.2cm}
        $L = 18.3 \ \unit{cm}$
        \vspace{0.3cm}
        \label{my-label}
        \begin{tabular}{c|c}
        $d_1 \unit{\ (cm)}$            & $d_2 \unit{\ (cm)} \pm 1.5$ \\ \hline
        13.9                       & 37.3                    \\
        13.9                       & 34.0                    \\
        13.9                       & 39.2                    \\
        13.9                       & 37.5                    \\
        13.9                       & 35.6                    \\
        13.9                       & 37.2
        \end{tabular}
    \end{minipage}
\end{table}

%
% \begin{figure}
%     \begin{minipage}{.45\textwidth}
%         \includegraphics[width=\textwidth]{path}
%         \caption
%     \end{minipage}
%     %
%     \begin{minipage}{.45\textwidth}
%
%     \end{minipage}
% \end{figure}
\section{Questions}

\begin{enumerate}
\item {\textit{How far away does the source have to be before its presence makes no difference?}
\begin{quote}
    Not very far. Half a meter at most. We kept it on the other end of the table.
\end{quote}}

\item{\textit{How does the strength of the source fall off with distance? Discuss.}
\begin{quote}
    Inverse-exponentially. If we assume that the source emits beta particles uniformly in all directions, we would imagine that the strength of the source falls off with the distance square, times four pi.
\end{quote}}

\item{\textit{How does the intensity of the light vary in the different materials?}
\begin{quote}
    $I = \dfrac{cn\epsilon_0}{2}E^2$, so the intensity of light increases with the refractive index of the medium it travels in.
\end{quote}}

\end{enumerate}

\section{Analysis}
The uncertainty of our measurement for the index of refraction of air is produced by our error in the distance between the reflector and the laser source. It is equal to 0.016 and is unitless. Our expected value for the index of refraction of air (which of course depends on the density of the air, which depends on the ambient temperature and pressure) is 1.00, which falls within our estimated uncertainty. \\
The speed of light in our first medium, $\text{H}_2\text{O}$ is slightly higher than expected, producting an index of refraction value of $1.39$. The expected value for the refractive index of water is 1.33, but of course this value is dependent on temperatue, a quantity that we did not measure. However, the water was fairly cold, which is consistent with the increased index of refraction that we observed. Another factor that likely contributed to this high index of refraction was the fact that our water was contained in a tube bounded by two glass discs. The assumption was that we could ignore them because their thickness was substantially smalled than the length of water that the laser passed through, but because the index of refraction of glass is likely higher than that for water, their presence would likely increase the value of our measurement.\\
The index of refraction of our second medium, ``acryliglas", was reported to 1.49. This is consistent with values that I found on line, but is likely a bit high given our assumption that the index of refraction of air was one.\\
The index of refraction of our third medium was 2.25, which is much higher than one would expect for acrylic, but the rod was reportedly full of impurities, which would have the effect of increasing the measured refractive index.



\end{document}
